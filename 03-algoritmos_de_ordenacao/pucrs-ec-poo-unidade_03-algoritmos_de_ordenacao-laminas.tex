\documentclass[aspectratio=169]{beamer}
\usepackage[utf8]{inputenc}
\usepackage[T1]{fontenc}
\usepackage[brazil]{babel}
\usepackage{ragged2e}
\usepackage{booktabs}
\usepackage{verbatim}
\usetheme{AnnArbor}
\usecolortheme{orchid}
\usefonttheme[onlymath]{serif}
\usepackage{listings}
\usepackage{colortbl	}
\usepackage{array}

\newcolumntype{C}[0]{>{\centering\arraybackslash}p{0.4cm}}

\lstset{language=C++,
	backgroundcolor=\color{green!10},
	basicstyle=\ttfamily,
	keywordstyle=\color{blue}\ttfamily,
	stringstyle=\color{red}\ttfamily,
	commentstyle=\color{green}\ttfamily,
	morecomment=[l][\color{magenta}]{\#}
}

\AtBeginSection[]{
  \begin{frame}
  \vfill
  \centering
  \begin{beamercolorbox}[sep=8pt,center,shadow=true,rounded=true]{title}
    \usebeamerfont{title}\insertsectionhead\par%
  \end{beamercolorbox}
  \vfill
  \end{frame}
}

\title[\sc{Algoritmos de Ordenação}]{Algoritmos de Ordenação}
\author[Roland Teodorowitsch]{Roland Teodorowitsch}
\institute[ALEST I - EP - PUCRS]{Algoritmos e Estruturas de Dados I - Escola Politécnica - PUCRS}
\date{18 de agosto de 2023}

\begin{document}
\justifying

%-------------------------------------------------------
\begin{frame}
	\titlepage
\end{frame}

%=======================================================
\section{Introdução}

%-------------------------------------------------------
\begin{frame}\frametitle{Leitura(s) Recomendada(s)}

\begin{columns}[T]
\begin{column}{0.15\linewidth}
\vspace{-3mm}
\begin{figure}[h]
	\centering
	\includegraphics[height=0.3\paperheight]{imagens/livro_goodrich.jpg}
\end{figure}
\end{column}
\begin{column}{0.85\linewidth}
\vspace{3mm}
\textbf{Seções 3.1.2, 11.1 (\emph{Merge Sort}), 11.2 (\emph{Quick Sort}), 11.3.3 (comparação)}\\
\scriptsize{GOODRICH, Michael T.; TAMASSIA, Roberto. \textbf{Estruturas de dados e algoritmos em Java}. Tradução: Bernardo Copstein. 5. ed. Porto Alegre: Bookman, 2013. xxii, 713 p. E-book. ISBN 9788582600191. Tradução de: Data Structures and Algorithms in Java, 5th Edition. Disponível em: \textless{}\url{https://integrada.minhabiblioteca.com.br/\#/books/9788582600191/}\textgreater{}. Acesso em: 01 ago. 2023.}
\end{column}
\end{columns}

\end{frame}

%-------------------------------------------------------
\begin{frame}\frametitle{Sites sobre Ordenação [*]}
\begin{itemize}
{\small
	\item Animações:\\\url{http://www.sorting-algorithms.com/}
	\item Algoritmos na wikipedia:\\\url {https://en.wikipedia.org/wiki/Sorting_algorithm}
	\item Danças:\\\url{http://makezine.com/2011/04/12/data-sorting-dances/}
	\item 15 algoritmos em 6 minutos:\\\url{https://www.youtube.com/watch?v=kPRA0W1kECg}
	\item Visualização e comparação de algoritmos de ordenação:\\\url{https://www.youtube.com/watch?v=ZZuD6iUe3Pc}
	\item Visualização \emph{Bubble Sort vs Quick Sort}:\\\url{https://www.youtube.com/watch?v=aXXWXz5rF64}
	\item Visualização \emph{Merge Sort vs Quick Sort}:\\\url{https://www.youtube.com/watch?v=es2T6KY45cA}
}
\end{itemize}
\end{frame}

%-------------------------------------------------------
\begin{frame}\frametitle{Revisão: Algoritmos de Pesquisa}
\begin{itemize}
	\item Pesquisa Linear
	\begin{itemize}
		\item Pode ser aplicada sobre qualquer coleção, ordenada ou não
		\item Procura um item, comparando-o com cada elemento da coleção, até achar ou chegar no final
		\item Melhor caso: o item procurado está na primeira posição da coleção
		\item Pior caso: o item NÃO está na coleção
		\item Complexidade: $O(n)$
	\end{itemize}
	\item Pesquisa Binária
	\begin{itemize}
		\item A coleção deve estar ordenada
		\item Estratégia básica:
		\begin{itemize}
			\item Verifica o elemento central: se encontrou, a busca termina
			\item Se o item for menor que o central, considera apenas a parte abaixo do elemento central
			\item Se o item for maior que o central, considera apenas a parte acima do elemento central
		\end{itemize}
		\item Trabalha subdividindo a coleção e reaplicando sempre a estratégia básica, o que o torna adequado para implementação recursiva
		\item Complexidade: $O(\log n)$
	\end{itemize}
\end{itemize}
\end{frame}

%=======================================================
\section{Algoritmos de Ordenação}

%-------------------------------------------------------
\begin{frame}\frametitle{Algoritmos de Ordenação}
\begin{itemize}
	\item Organizam os elementos de uma coleção segundo determinado critério (ordem crescente de valor, por exemplo)
	\item Operação básica: troca de elementos
	\item Exemplos
	\begin{itemize}
		\item \emph{Bubble Sort}
		\item \emph{Selection Sort}
		\item \emph{Insertion Sort}
		\item \emph{Merge Sort}
		\item \emph{Quick Sort}
		\item etc.
	\end{itemize}
	\item Em geral, os mais simples nem sempre tem bom desempenho (menos otimizados)
	\item Algoritmos com bom desempenho costumam ser mais sofisticados
	\item São importantes quando se quer implementar busca eficiente (pesquisa binária)
\end{itemize}
\end{frame}

%=======================================================
\section{\emph{Bubble Sort}}

%-------------------------------------------------------
\begin{frame}\frametitle{\emph{Bubble Sort}}
\begin{itemize}
	\item É um dos métodos mais simples de ordenação
	\item Estratégia: compara elementos adjacentes, e, se estiverem fora de ordem, troca os elementos
	\item Repete-se a estratégia básica até que a coleção esteja ordenada
	\item Complexidade: $O(n)$ (melhor caso) ou $O(n^2)$ (pior caso)
\end{itemize}
\end{frame}

%-------------------------------------------------------
\begin{frame}\frametitle{\emph{Bubble Sort}: Exemplo}
\begin{center}
\begin{tabular}{CCCCCCCCCC}
\tiny{0} & \tiny{1} & \tiny{2} & \tiny{3} & \tiny{4} & \tiny{5} & \tiny{6} & \tiny{7} & \tiny{8} & \tiny{9}\\
\hline
\multicolumn{1}{|C|}{5} & \multicolumn{1}{|C|}{7} & \multicolumn{1}{|C|}{8} & \multicolumn{1}{|C|}{1} & \multicolumn{1}{|C|}{10} & \multicolumn{1}{|C|}{9} & \multicolumn{1}{|C|}{4} & \multicolumn{1}{|C|}{6} & \multicolumn{1}{|C|}{3} & \multicolumn{1}{|C|}{2}\\
\hline
\hline
\multicolumn{1}{|C|}{5} & \multicolumn{1}{|C|}{7} & \multicolumn{1}{|C|}{1} & \multicolumn{1}{|C|}{8} & \multicolumn{1}{|C|}{9} & \multicolumn{1}{|C|}{4} & \multicolumn{1}{|C|}{6} & \multicolumn{1}{|C|}{3} & \multicolumn{1}{|C|}{2} & \multicolumn{1}{|C|}{\cellcolor{green}10}\\
\hline
\hline
\multicolumn{1}{|C|}{5} & \multicolumn{1}{|C|}{1} & \multicolumn{1}{|C|}{7} & \multicolumn{1}{|C|}{8} & \multicolumn{1}{|C|}{4} & \multicolumn{1}{|C|}{6} & \multicolumn{1}{|C|}{3} & \multicolumn{1}{|C|}{2} & \multicolumn{1}{|C|}{\cellcolor{green}9} & \multicolumn{1}{|C|}{\cellcolor{green}10}\\
\hline
\hline
\multicolumn{1}{|C|}{1} & \multicolumn{1}{|C|}{5} & \multicolumn{1}{|C|}{7} & \multicolumn{1}{|C|}{4} & \multicolumn{1}{|C|}{6} & \multicolumn{1}{|C|}{3} & \multicolumn{1}{|C|}{2} & \multicolumn{1}{|C|}{\cellcolor{green}8} & \multicolumn{1}{|C|}{\cellcolor{green}9} & \multicolumn{1}{|C|}{\cellcolor{green}10}\\
\hline
\hline
\multicolumn{1}{|C|}{1} & \multicolumn{1}{|C|}{5} & \multicolumn{1}{|C|}{4} & \multicolumn{1}{|C|}{6} & \multicolumn{1}{|C|}{3} & \multicolumn{1}{|C|}{2} & \multicolumn{1}{|C|}{\cellcolor{green}7} & \multicolumn{1}{|C|}{\cellcolor{green}8} & \multicolumn{1}{|C|}{\cellcolor{green}9} & \multicolumn{1}{|C|}{\cellcolor{green}10}\\
\hline
\hline
\multicolumn{1}{|C|}{1} & \multicolumn{1}{|C|}{4} & \multicolumn{1}{|C|}{5} & \multicolumn{1}{|C|}{3} & \multicolumn{1}{|C|}{2} & \multicolumn{1}{|C|}{\cellcolor{green}6} & \multicolumn{1}{|C|}{\cellcolor{green}7} & \multicolumn{1}{|C|}{\cellcolor{green}8} & \multicolumn{1}{|C|}{\cellcolor{green}9} & \multicolumn{1}{|C|}{\cellcolor{green}10}\\
\hline
\hline
\multicolumn{1}{|C|}{1} & \multicolumn{1}{|C|}{4} & \multicolumn{1}{|C|}{3} & \multicolumn{1}{|C|}{2} & \multicolumn{1}{|C|}{\cellcolor{green}5} & \multicolumn{1}{|C|}{\cellcolor{green}6} & \multicolumn{1}{|C|}{\cellcolor{green}7} & \multicolumn{1}{|C|}{\cellcolor{green}8} & \multicolumn{1}{|C|}{\cellcolor{green}9} & \multicolumn{1}{|C|}{\cellcolor{green}10}\\
\hline
\hline
\multicolumn{1}{|C|}{1} & \multicolumn{1}{|C|}{3} & \multicolumn{1}{|C|}{2} & \multicolumn{1}{|C|}{\cellcolor{green}4} & \multicolumn{1}{|C|}{\cellcolor{green}5} & \multicolumn{1}{|C|}{\cellcolor{green}6} & \multicolumn{1}{|C|}{\cellcolor{green}7} & \multicolumn{1}{|C|}{\cellcolor{green}8} & \multicolumn{1}{|C|}{\cellcolor{green}9} & \multicolumn{1}{|C|}{\cellcolor{green}10}\\
\hline
\hline
\multicolumn{1}{|C|}{1} & \multicolumn{1}{|C|}{2} & \multicolumn{1}{|C|}{\cellcolor{green}3} & \multicolumn{1}{|C|}{\cellcolor{green}4} & \multicolumn{1}{|C|}{\cellcolor{green}5} & \multicolumn{1}{|C|}{\cellcolor{green}6} & \multicolumn{1}{|C|}{\cellcolor{green}7} & \multicolumn{1}{|C|}{\cellcolor{green}8} & \multicolumn{1}{|C|}{\cellcolor{green}9} & \multicolumn{1}{|C|}{\cellcolor{green}10}\\
\hline
\hline
\multicolumn{1}{|C|}{\cellcolor{green}1} & \multicolumn{1}{|C|}{\cellcolor{green}2} & \multicolumn{1}{|C|}{\cellcolor{green}3} & \multicolumn{1}{|C|}{\cellcolor{green}4} & \multicolumn{1}{|C|}{\cellcolor{green}5} & \multicolumn{1}{|C|}{\cellcolor{green}6} & \multicolumn{1}{|C|}{\cellcolor{green}7} & \multicolumn{1}{|C|}{\cellcolor{green}8} & \multicolumn{1}{|C|}{\cellcolor{green}9} & \multicolumn{1}{|C|}{\cellcolor{green}10}\\
\hline
\end{tabular}
\end{center}
\end{frame}	

%-------------------------------------------------------
\begin{frame}[fragile]\frametitle{\emph{Bubble Sort}: Implementação}
\lstinputlisting[basicstyle=\ttfamily\small]{src/bubbleSort.cpp}
\end{frame}

%-------------------------------------------------------
\begin{frame}[fragile]\frametitle{\emph{Bubble Sort}: Mais informações [*]}
\begin{itemize}
	\item \url{http://www.sorting-algorithms.com/bubble-sort}
	\item \url{https://www.hackerearth.com/practice/algorithms/sorting/bubble-sort/tutorial/}
\end{itemize}
\end{frame}

%=======================================================
\section{\emph{Selection Sort}}

%-------------------------------------------------------
\begin{frame}\frametitle{\emph{Selection Sort}}
\begin{itemize}
	\item É um algorimo de ordenação por seleção
	\item Fácil de implementar e bastante intuitivo, o que não garante eficiência...
	\item Estratégia: procurar o menor elemento e colocá-lo na sua posiçao
	\item Repete-se a estratégia até que todos os elementos estejam em sua posição
	\item Complexidade: $O(n^2)$ (melhor e pior caso)
\end{itemize}
\end{frame}

%-------------------------------------------------------
\begin{frame}\frametitle{\emph{Selection Sort}: Exemplo}
\begin{center}
\begin{tabular}{CCCCCCCCCC}
\tiny{0} & \tiny{1} & \tiny{2} & \tiny{3} & \tiny{4} & \tiny{5} & \tiny{6} & \tiny{7} & \tiny{8} & \tiny{9}\\
\hline
\multicolumn{1}{|C|}{5} & \multicolumn{1}{|C|}{7} & \multicolumn{1}{|C|}{8} & \multicolumn{1}{|C|}{1} & \multicolumn{1}{|C|}{10} & \multicolumn{1}{|C|}{9} & \multicolumn{1}{|C|}{4} & \multicolumn{1}{|C|}{6} & \multicolumn{1}{|C|}{3} & \multicolumn{1}{|C|}{2}\\
\hline
\hline
\multicolumn{1}{|C|}{\cellcolor{green}1} & \multicolumn{1}{|C|}{7} & \multicolumn{1}{|C|}{8} & \multicolumn{1}{|C|}{5} & \multicolumn{1}{|C|}{10} & \multicolumn{1}{|C|}{9} & \multicolumn{1}{|C|}{4} & \multicolumn{1}{|C|}{6} & \multicolumn{1}{|C|}{3} & \multicolumn{1}{|C|}{2}\\
\hline
\hline
\multicolumn{1}{|C|}{\cellcolor{green}1} & \multicolumn{1}{|C|}{\cellcolor{green}2} & \multicolumn{1}{|C|}{8} & \multicolumn{1}{|C|}{5} & \multicolumn{1}{|C|}{10} & \multicolumn{1}{|C|}{9} & \multicolumn{1}{|C|}{4} & \multicolumn{1}{|C|}{6} & \multicolumn{1}{|C|}{3} & \multicolumn{1}{|C|}{7}\\
\hline
\hline
\multicolumn{1}{|C|}{\cellcolor{green}1} & \multicolumn{1}{|C|}{\cellcolor{green}2} & \multicolumn{1}{|C|}{\cellcolor{green}3} & \multicolumn{1}{|C|}{5} & \multicolumn{1}{|C|}{10} & \multicolumn{1}{|C|}{9} & \multicolumn{1}{|C|}{4} & \multicolumn{1}{|C|}{6} & \multicolumn{1}{|C|}{8} & \multicolumn{1}{|C|}{7}\\
\hline
\hline
\multicolumn{1}{|C|}{\cellcolor{green}1} & \multicolumn{1}{|C|}{\cellcolor{green}2} & \multicolumn{1}{|C|}{\cellcolor{green}3} & \multicolumn{1}{|C|}{\cellcolor{green}4} & \multicolumn{1}{|C|}{10} & \multicolumn{1}{|C|}{9} & \multicolumn{1}{|C|}{5} & \multicolumn{1}{|C|}{6} & \multicolumn{1}{|C|}{8} & \multicolumn{1}{|C|}{7}\\
\hline
\hline
\multicolumn{1}{|C|}{\cellcolor{green}1} & \multicolumn{1}{|C|}{\cellcolor{green}2} & \multicolumn{1}{|C|}{\cellcolor{green}3} & \multicolumn{1}{|C|}{\cellcolor{green}4} & \multicolumn{1}{|C|}{\cellcolor{green}5} & \multicolumn{1}{|C|}{9} & \multicolumn{1}{|C|}{10} & \multicolumn{1}{|C|}{6} & \multicolumn{1}{|C|}{8} & \multicolumn{1}{|C|}{7}\\
\hline
\hline
\multicolumn{1}{|C|}{\cellcolor{green}1} & \multicolumn{1}{|C|}{\cellcolor{green}2} & \multicolumn{1}{|C|}{\cellcolor{green}3} & \multicolumn{1}{|C|}{\cellcolor{green}4} & \multicolumn{1}{|C|}{\cellcolor{green}5} & \multicolumn{1}{|C|}{\cellcolor{green}6} & \multicolumn{1}{|C|}{10} & \multicolumn{1}{|C|}{9} & \multicolumn{1}{|C|}{8} & \multicolumn{1}{|C|}{7}\\
\hline
\hline
\multicolumn{1}{|C|}{\cellcolor{green}1} & \multicolumn{1}{|C|}{\cellcolor{green}2} & \multicolumn{1}{|C|}{\cellcolor{green}3} & \multicolumn{1}{|C|}{\cellcolor{green}4} & \multicolumn{1}{|C|}{\cellcolor{green}5} & \multicolumn{1}{|C|}{\cellcolor{green}6} & \multicolumn{1}{|C|}{\cellcolor{green}7} & \multicolumn{1}{|C|}{9} & \multicolumn{1}{|C|}{8} & \multicolumn{1}{|C|}{10}\\
\hline
\hline
\multicolumn{1}{|C|}{\cellcolor{green}1} & \multicolumn{1}{|C|}{\cellcolor{green}2} & \multicolumn{1}{|C|}{\cellcolor{green}3} & \multicolumn{1}{|C|}{\cellcolor{green}4} & \multicolumn{1}{|C|}{\cellcolor{green}5} & \multicolumn{1}{|C|}{\cellcolor{green}6} & \multicolumn{1}{|C|}{\cellcolor{green}7} & \multicolumn{1}{|C|}{\cellcolor{green}8} & \multicolumn{1}{|C|}{9} & \multicolumn{1}{|C|}{10}\\
\hline
\hline
\multicolumn{1}{|C|}{\cellcolor{green}1} & \multicolumn{1}{|C|}{\cellcolor{green}2} & \multicolumn{1}{|C|}{\cellcolor{green}3} & \multicolumn{1}{|C|}{\cellcolor{green}4} & \multicolumn{1}{|C|}{\cellcolor{green}5} & \multicolumn{1}{|C|}{\cellcolor{green}6} & \multicolumn{1}{|C|}{\cellcolor{green}7} & \multicolumn{1}{|C|}{\cellcolor{green}8} & \multicolumn{1}{|C|}{\cellcolor{green}9} & \multicolumn{1}{|C|}{\cellcolor{green}10}\\
\hline
\end{tabular}
\end{center}
\end{frame}

%-------------------------------------------------------
\begin{frame}\frametitle{\emph{Selection Sort}: Implementação}
\lstinputlisting[basicstyle=\ttfamily\small]{src/selectionSort.cpp}
\end{frame}

%=======================================================
\section{\emph{Insertion Sort}}

%-------------------------------------------------------
\begin{frame}\frametitle{\emph{Insertion Sort}}
\begin{itemize}
	\item É um algorimo de ordenação por inserção
	\item Estratégia:
	\begin{itemize}
		\item Escolhe-se uma base que inicia no segundo elemento e avança até o último elemento
		\item Sempre à esquerda da base todos os elementos devem estar ordenados
		\item Busca-se a posição da base nos elementos à esquerda, sempre deslocando os elementos uma posição para a direita enquanto não chegar na posição correta da base
		\item Quando chegar na posição correta da base, atribui-se o valor da base para esta posição
	\end{itemize}
	\item Trata-se de uma algoritmo um pouco mais avançado do que os dois anteriores
	\item Complexidade: $O(n)$ (melhor caso) ou $O(n^2)$ (pior caso)
\end{itemize}
\end{frame}	

%-------------------------------------------------------
\begin{frame}\frametitle{\emph{Insertion Sort}: Exemplo 1}
\begin{figure}[h]
	\centering
	\includegraphics[height=0.6\paperheight]{imagens/insertion_sort.png}\\
	~ {\tiny Fonte: \url{https://www.geeksforgeeks.org/insertion-sort/}}
\end{figure}
\end{frame}

%-------------------------------------------------------
\begin{frame}\frametitle{\emph{Insertion Sort}: Exemplo 2}
\begin{center}
\begin{tabular}{CCCCCCCCCC}
\tiny{0} & \tiny{1} & \tiny{2} & \tiny{3} & \tiny{4} & \tiny{5} & \tiny{6} & \tiny{7} & \tiny{8} & \tiny{9}\\
\hline
\multicolumn{1}{|C|}{\cellcolor{lime}5} & \multicolumn{1}{|C|}{\cellcolor{yellow}7} & \multicolumn{1}{|C|}{8} & \multicolumn{1}{|C|}{1} & \multicolumn{1}{|C|}{10} & \multicolumn{1}{|C|}{9} & \multicolumn{1}{|C|}{4} & \multicolumn{1}{|C|}{6} & \multicolumn{1}{|C|}{3} & \multicolumn{1}{|C|}{2}\\
\hline
\hline
\multicolumn{1}{|C|}{\cellcolor{lime}5} & \multicolumn{1}{|C|}{\cellcolor{lime}7} & \multicolumn{1}{|C|}{\cellcolor{yellow}8} & \multicolumn{1}{|C|}{1} & \multicolumn{1}{|C|}{10} & \multicolumn{1}{|C|}{9} & \multicolumn{1}{|C|}{4} & \multicolumn{1}{|C|}{6} & \multicolumn{1}{|C|}{3} & \multicolumn{1}{|C|}{2}\\
\hline
\hline
\multicolumn{1}{|C|}{\cellcolor{lime}5} & \multicolumn{1}{|C|}{\cellcolor{lime}7} & \multicolumn{1}{|C|}{\cellcolor{lime}8} & \multicolumn{1}{|C|}{\cellcolor{yellow}1} & \multicolumn{1}{|C|}{10} & \multicolumn{1}{|C|}{9} & \multicolumn{1}{|C|}{4} & \multicolumn{1}{|C|}{6} & \multicolumn{1}{|C|}{3} & \multicolumn{1}{|C|}{2}\\
\hline
\hline
\multicolumn{1}{|C|}{\cellcolor{lime}1} & \multicolumn{1}{|C|}{\cellcolor{lime}5} & \multicolumn{1}{|C|}{\cellcolor{lime}7} & \multicolumn{1}{|C|}{\cellcolor{lime}8} & \multicolumn{1}{|C|}{\cellcolor{yellow}10} & \multicolumn{1}{|C|}{9} & \multicolumn{1}{|C|}{4} & \multicolumn{1}{|C|}{6} & \multicolumn{1}{|C|}{3} & \multicolumn{1}{|C|}{2}\\
\hline
\hline
\multicolumn{1}{|C|}{\cellcolor{lime}1} & \multicolumn{1}{|C|}{\cellcolor{lime}5} & \multicolumn{1}{|C|}{\cellcolor{lime}7} & \multicolumn{1}{|C|}{\cellcolor{lime}8} & \multicolumn{1}{|C|}{\cellcolor{lime}10} & \multicolumn{1}{|C|}{\cellcolor{yellow}9} & \multicolumn{1}{|C|}{4} & \multicolumn{1}{|C|}{6} & \multicolumn{1}{|C|}{3} & \multicolumn{1}{|C|}{2}\\
\hline
\hline
\multicolumn{1}{|C|}{\cellcolor{lime}1} & \multicolumn{1}{|C|}{\cellcolor{lime}5} & \multicolumn{1}{|C|}{\cellcolor{lime}7} & \multicolumn{1}{|C|}{\cellcolor{lime}8} & \multicolumn{1}{|C|}{\cellcolor{lime}9} & \multicolumn{1}{|C|}{\cellcolor{lime}10} & \multicolumn{1}{|C|}{\cellcolor{yellow}4} & \multicolumn{1}{|C|}{6} & \multicolumn{1}{|C|}{3} & \multicolumn{1}{|C|}{2}\\
\hline
\hline
\multicolumn{1}{|C|}{\cellcolor{lime}1} & \multicolumn{1}{|C|}{\cellcolor{lime}4} & \multicolumn{1}{|C|}{\cellcolor{lime}5} & \multicolumn{1}{|C|}{\cellcolor{lime}7} & \multicolumn{1}{|C|}{\cellcolor{lime}8} & \multicolumn{1}{|C|}{\cellcolor{lime}9} & \multicolumn{1}{|C|}{\cellcolor{lime}10} & \multicolumn{1}{|C|}{\cellcolor{yellow}6} & \multicolumn{1}{|C|}{3} & \multicolumn{1}{|C|}{2}\\
\hline
\hline
\multicolumn{1}{|C|}{\cellcolor{lime}1} & \multicolumn{1}{|C|}{\cellcolor{lime}4} & \multicolumn{1}{|C|}{\cellcolor{lime}5} & \multicolumn{1}{|C|}{\cellcolor{lime}6} & \multicolumn{1}{|C|}{\cellcolor{lime}7} & \multicolumn{1}{|C|}{\cellcolor{lime}8} & \multicolumn{1}{|C|}{\cellcolor{lime}9} & \multicolumn{1}{|C|}{\cellcolor{lime}10} & \multicolumn{1}{|C|}{\cellcolor{yellow}3} & \multicolumn{1}{|C|}{2}\\
\hline
\hline
\multicolumn{1}{|C|}{\cellcolor{lime}1} & \multicolumn{1}{|C|}{\cellcolor{lime}3} & \multicolumn{1}{|C|}{\cellcolor{lime}4} & \multicolumn{1}{|C|}{\cellcolor{lime}5} & \multicolumn{1}{|C|}{\cellcolor{lime}6} & \multicolumn{1}{|C|}{\cellcolor{lime}7} & \multicolumn{1}{|C|}{\cellcolor{lime}8} & \multicolumn{1}{|C|}{\cellcolor{lime}9} & \multicolumn{1}{|C|}{\cellcolor{lime}10} & \multicolumn{1}{|C|}{\cellcolor{yellow}2}\\
\hline
\hline
\multicolumn{1}{|C|}{\cellcolor{green}1} & \multicolumn{1}{|C|}{\cellcolor{green}2} & \multicolumn{1}{|C|}{\cellcolor{green}3} & \multicolumn{1}{|C|}{\cellcolor{green}4} & \multicolumn{1}{|C|}{\cellcolor{green}5} & \multicolumn{1}{|C|}{\cellcolor{green}6} & \multicolumn{1}{|C|}{\cellcolor{green}7} & \multicolumn{1}{|C|}{\cellcolor{green}8} & \multicolumn{1}{|C|}{\cellcolor{green}9} & \multicolumn{1}{|C|}{\cellcolor{green}10}\\
\hline
\end{tabular}
\end{center}
\end{frame}

%-------------------------------------------------------
\begin{frame}\frametitle{\emph{Insertion Sort}: Implementação}
\lstinputlisting{src/insertionSort.cpp}
\end{frame}

%-------------------------------------------------------
\begin{frame}[fragile]\frametitle{\emph{Insertion Sort}: Mais informações [*]}
\begin{itemize}
	\item \url{http://www.sorting-algorithms.com/insertion-sort}
	\item \url{https://www.hackerearth.com/practice/algorithms/sorting/insertion-sort/tutorial/}
\end{itemize}
\end{frame}

%=======================================================
\section{\emph{Merge Sort}}

%-------------------------------------------------------
\begin{frame}\frametitle{\emph{Merge Sort}}
\begin{itemize}
	\item É um algorimo de ordenação por intercalação
	\item Utiliza o padrão (estratégia) conhecido como ``divisão e conquista''
	\item Consiste de 3 etapas
	\begin{itemize}
		\item Divisão: se há algo a ordenar, divide os dados de entrada em duas (ou mais) partes e executa o algoritmo sobre cada uma das partes; se não há nada a ordenar, retorna a solução
		\item Conquista: cada parte dos dados é classificada recursivamente
		\item Combinação: quando cada subconjunto está classificado (internamente), eles devem ser combinados (\emph{merge}) realizando-se uma intercalação
	\end{itemize}
	\item Permite implementação recursiva
\end{itemize}
\end{frame}

%-------------------------------------------------------
\begin{frame}\frametitle{\emph{Merge Sort}: Estratégia [*]}
\begin{itemize}
	\item  Para ordenar uma sequência $S$ com $n$ elementos:
	\begin{itemize}
		\item \textbf{Dividir}: se $S$ tem zero ou um elemento, retorna $S$, pois já está classificado; senão, remove os elementos de $S$ e coloca-os em duas sequências, $S_1$ e $S_2$ ($n/2$ elementos em cada um)
		\item \textbf{Conquistar}: classifica as sequências $S_1$ e $S_2$ recursivamente
		\item \textbf{Combinar}: coloca os elementos de volta em $S$ com a união das sequências $S_1$ e $S_2$ ordenadas
	\end{itemize}
	\end{itemize}
\end{frame}

%-------------------------------------------------------
\begin{frame}\frametitle{\emph{Merge Sort}: Exemplo}
\begin{columns}[T]
\begin{column}{0.6\linewidth}
\vspace{-3mm}
\begin{figure}[h]
	\centering
	\includegraphics[height=0.7\paperheight]{imagens/mergesort.png}\\
	\tiny{Fonte: \url{https://en.wikipedia.org/wiki/Merge_sort}}
\end{figure}
\end{column}
\begin{column}{0.4\linewidth}
\vspace{3mm}
\begin{itemize}
	\item A execução do algoritmo pode ser vista como uma árvore binária
	\item Cada nodo representa uma chamada recursiva do algoritmo \emph{Merge Sort}
	\item Nodos recebem sequências de entrada para serem processadas e, por fim, geram sequências de saída ordenadas
\end{itemize}
\end{column}
\end{columns}
\end{frame}

%-------------------------------------------------------
\begin{frame}\frametitle{\emph{Merge Sort}: Implementação}
\lstinputlisting[basicstyle=\ttfamily\scriptsize]{src/mergeSort.cpp}
\end{frame}

%-------------------------------------------------------
\begin{frame}\frametitle{\emph{Merge Sort}: Desempenho [*]}
\begin{columns}[T]
\begin{column}{0.4\linewidth}
\vspace{3mm}
\begin{itemize}
	\item O tamanho da sequência de entrada é a metade a cada chamada recursiva
	\item A árvore associada a uma execução do algoritmo com uma sequência de tamanho $n$, tem altura $\log n$
	\item Conclusões:
	\begin{itemize}
		\item Altura da árvore é $\log n$
		\item Tempo gasto em cada nível: $O(n)$
		\item Tempo de execução: $O(n\log n)$
	\end{itemize}
\end{itemize}
\end{column}
\begin{column}{0.6\linewidth}
\vspace{-3mm}
\begin{figure}[h]
	\centering
	\includegraphics[height=0.6\paperheight]{imagens/mergesort_desempenho.png}
\end{figure}
\end{column}
\end{columns}
\end{frame}

%-------------------------------------------------------
\begin{frame}[fragile]\frametitle{\emph{Merge Sort}: Mais informações [*]}
\begin{itemize}
	\item \url{http://www.sorting-algorithms.com/merge-sort}
	\item \url{https://www.hackerearth.com/practice/algorithms/sorting/merge-sort/tutorial/}
\end{itemize}
\end{frame}

%=======================================================
\section{\emph{Quick Sort}}

%-------------------------------------------------------
\begin{frame}\frametitle{\emph{Quick Sort}}
\begin{itemize}	
	\item ...
\end{itemize}
%	\lstinputlisting[basicstyle=\ttfamily\scriptsize]{src/quickSort.cpp}
\end{frame}

%=======================================================
\section{Comparação}

%-------------------------------------------------------
\begin{frame}\frametitle{Comparação}
\begin{itemize}	
	\item ...
\end{itemize}
%	\lstinputlisting[basicstyle=\ttfamily\scriptsize]{src/quickSort.cpp}
\end{frame}

\begin{comment}



%-------------------------------------------------------
\begin{frame}\frametitle{Exemplo clássico: Fatorial}
\begin{itemize}
	\item O fatorial de n é normalmente definido como:
\[ n! = \prod_{k=1}^{n}{k} = n \times (n-1) \times (n-2) \times ... \times 3 \times 2 \times 1,\qquad \forall n \in \mathbb{N}\]
	\item Por exemplo:
\[5 ! = 5 \times 4 \times 3 \times 2 \times 1 = 120\]
	\item Mas também pode-se usar a sua definição recursiva:
\[ n! = \begin{cases}
1, & \mbox{se ~ } n=0 \mbox{~ ou ~ } n=1\\
n \times (n-1)!, & \mbox{se ~ } n > 1
\end{cases} \]
\end{itemize}
\end{frame}

%-------------------------------------------------------
\begin{frame}[fragile]\frametitle{Implementação do Fatorial}
\begin{itemize}
	\item Iterativo
{\scriptsize\lstinputlisting{src/fatorial.cpp}}
	\item Recursivo
{\scriptsize\lstinputlisting{src/fatorialRec.cpp}}
\end{itemize}
\end{frame}

%-------------------------------------------------------
\begin{frame}\frametitle{Exemplo: Contagem}
\begin{itemize}
	\item Iterativo
{\scriptsize\lstinputlisting{src/contagem.cpp}}
	\pause
	\item Recursivo
{\scriptsize\lstinputlisting{src/contagemRec.cpp}}
\end{itemize}
\end{frame}

%-------------------------------------------------------
\begin{frame}\frametitle{Exercício: Contagem Regressiva}
\begin{itemize}
	\item Iterativo
{\scriptsize\lstinputlisting{src/contagemRegressiva.cpp}}
	\pause
	\item Recursivo
{\scriptsize\lstinputlisting{src/contagemRegressivaRec.cpp}}
\end{itemize}
\end{frame}

%-------------------------------------------------------
\begin{frame}\frametitle{Exercício: Geração de Índices de uma Matriz}
\begin{itemize}
	\item Iterativo
{\scriptsize\lstinputlisting{src/indicesMatriz.cpp}}
	\pause
	\item Recursivo
{\scriptsize\lstinputlisting{src/indicesMatrizRec1.cpp}}
	\pause
	\item \textbf{Desafio:} faça uma versão recursiva do algoritmo acima sem usar nenhum laço!
\end{itemize}
\end{frame}

%-------------------------------------------------------
\begin{frame}\frametitle{Recursividade [*]}
\begin{itemize}
	\item Poderosa ferramenta de programação
	\item Apesar de bastante empregada, nem sempre ela deve ser aplicada
	\begin{itemize}
		\item É preciso analisar o problema e ver se necessita de uma solução recursiva
	\end{itemize}
	\item Quando bem empregada pode tornar a solução de um problema clara, simples e consisa
\end{itemize}
\end{frame}

%-------------------------------------------------------
\begin{frame}\frametitle{Vantagens [*]}
\begin{itemize}
	\item Rotinas mais concisas
	\item Relação direta com uma prova por indução matemática
	\begin{itemize}
		\item Indução matemática: metodo de prova matemática usado para demonstrar a verdade de um número infinito de proposições
		\begin{itemize}
			\item Válida se funciona para $n$ igual a $0$ ou $1$
			\item Válida se vale para $n$ igual a $k$ e $k + 1$
		\end{itemize}
		\item Facilita verificar a correção
	\end{itemize}
\end{itemize}
\end{frame}

%-------------------------------------------------------
\begin{frame}\frametitle{Desvantagens [*]}
\begin{itemize}
	\item Cada chamada recursiva implica em um custo (tempo e espaço)
	\begin{itemize}
		\item Informações são armazenadas na pilha
		\item Para cada chamada realizada, um conjunto de variáveis locais é alocado (criado)
		\item Cada chamada requer
		\begin{itemize}
			\item O empilhamento de parâmetros e endereços de retorno da função que chama
			\item A alocação de variáveis locais da função % [Roland]
			%\item O desempilhamento de parâmetros pela função que executa
		\end{itemize}
		\item Cada retorno requer
		\begin{itemize}
			\item A desalocação das variáveis locais da função % [Roland]
			\item O desempilhamento do endereço de retorno
			%\item O empilhamento do resultado (ou passagem por registrador)
			%\item O desempilhamento do retorno pela função que chamou
		\end{itemize}
	\end{itemize}
\end{itemize}
\end{frame}

%=======================================================
\section{Algoritmos de Pesquisa}

%-------------------------------------------------------
\begin{frame}\frametitle{Algoritmos de Pesquisa}
\begin{itemize}
	\item Localizam um elemento dentro de uma coleção
	\item Podem retornar
	\begin{itemize}
		\item \textbf{Valor booleano} (\texttt{true} se o elemento existir na coleção ou \texttt{false}, em caso contrário) ou
		\item \textbf{Índice} (posição) do elemento na coleção (ou -1 se não encontrar)
	\end{itemize}
	\item Para coleções \textbf{desordenadas}, deve-se usar um algoritmo de \textbf{pesquisa linear}
	\item Quando a coleção está \textbf{ordenada}, pode-se usar um algoritmo de \textbf{pesquisa binária}
\end{itemize}
\end{frame}

%-------------------------------------------------------
\begin{frame}\frametitle{Pesquisa Linear}
\begin{itemize}
	\item Serve para coleções \textbf{desordenadas}
	\item Estratégia: comparar o elemento procurado com cada item da coleção até encontrar ou até chegar ao fim
	\item Melhor caso: o elemento procurado é o primeiro da lista
	\item Pior caso: o elemento procurado \textbf{NÃO} existe na lista
	\item Também poderia ser aplicado sobre coleções ordenadas
	\pause
	\item Complexidade: $O(n)$
\end{itemize}
\end{frame}

%-------------------------------------------------------
\begin{frame}[fragile]\frametitle{Implementação da Pesquisa Linear}
\begin{itemize}
	\item Iterativo
{\scriptsize\lstinputlisting{src/pesquisaLinear.cpp}}
	\pause
	\item Recursivo
{\scriptsize\lstinputlisting{src/pesquisaLinearRec.cpp}}
\end{itemize}
\end{frame}

%-------------------------------------------------------
\begin{frame}\frametitle{Pesquisa Binária}
\begin{itemize}
	\item Serve \textbf{APENAS} para coleções \textbf{ordenadas}
	\item Estratégia:
	\begin{itemize}
		\item Compara o valor procurado com o elemento do meio
		\begin{itemize}
			\item Se for igual, encontrou
			\item Se for menor, continua-se a busca na metade inferior (desconsidera a metade superior)
			\item Se for maior, continua-se a busca na metade superior (desconsidera a metade inferior)
		\end{itemize}
		\item Repete-se o procecimento até que o valor procurado seja encontrado ou até que não se consiga dividir a coleção
	\end{itemize}
	\pause
		\item Complexidade: $O(\log{n})$
\end{itemize}
\end{frame}

%-------------------------------------------------------
\begin{frame}[fragile]\frametitle{Pesquisa Binária Iterativa}
\lstinputlisting[basicstyle=\ttfamily\scriptsize]{src/pesquisaBinaria.cpp}
\end{frame}

%-------------------------------------------------------
\begin{frame}\frametitle{Pesquisa Binária: Exemplo (\texttt{valor = 18})}
\begin{itemize}

	\item Passo 1:
\begin{tabular}{cccccccccc}
\tiny{0} & \tiny{1} & \tiny{2} & \tiny{3} & \tiny{4} & \tiny{5} & \tiny{6} & \tiny{7} & \tiny{8} & \tiny{9}\\
\hline
\multicolumn{1}{|c|}{2} & \multicolumn{1}{c|}{4} & \multicolumn{1}{c|}{6} & \multicolumn{1}{c|}{8} & \multicolumn{1}{c|}{10} & \multicolumn{1}{c|}{12} & \multicolumn{1}{c|}{14} & \multicolumn{1}{c|}{16} & \multicolumn{1}{c|}{18} & \multicolumn{1}{c|}{20}\\
\hline
$\uparrow$ & & & & $\uparrow$ & & & & & $\uparrow$ \\
{\tiny\texttt ini} & & & & {\tiny\texttt meio} & & & & & {\tiny\texttt fim} \\
\end{tabular}

	\item Passo 2:
\begin{tabular}{cccccccccc}
\tiny{0} & \tiny{1} & \tiny{2} & \tiny{3} & \tiny{4} & \tiny{5} & \tiny{6} & \tiny{7} & \tiny{8} & \tiny{9}\\
\hline
\multicolumn{1}{|c|}{2} & \multicolumn{1}{c|}{4} & \multicolumn{1}{c|}{6} & \multicolumn{1}{c|}{8} & \multicolumn{1}{c|}{10} & \multicolumn{1}{c|}{12} & \multicolumn{1}{c|}{14} & \multicolumn{1}{c|}{16} & \multicolumn{1}{c|}{18} & \multicolumn{1}{c|}{20}\\
\hline
 & & & & & $\uparrow$ & & $\uparrow$ & & $\uparrow$ \\
 & & & & & {\tiny\texttt ini} & & {\tiny\texttt meio} & & {\tiny\texttt fim} \\
\end{tabular}

	\item Passo 3:
\begin{tabular}{cccccccccc}
\tiny{0} & \tiny{1} & \tiny{2} & \tiny{3} & \tiny{4} & \tiny{5} & \tiny{6} & \tiny{7} & \tiny{8} & \tiny{9}\\
\hline
\multicolumn{1}{|c|}{2} & \multicolumn{1}{c|}{4} & \multicolumn{1}{c|}{6} & \multicolumn{1}{c|}{8} & \multicolumn{1}{c|}{10} & \multicolumn{1}{c|}{12} & \multicolumn{1}{c|}{14} & \multicolumn{1}{c|}{16} & \multicolumn{1}{c|}{18} & \multicolumn{1}{c|}{20}\\
\hline
 & & & & & & & & $\uparrow$ & $\uparrow$ \\
 & & & & & & & & {\tiny\texttt ini} & {\tiny\texttt fim} \\
 & & & & & & & & {\tiny\texttt meio} & \\
\end{tabular}
 ~ Valor 18 encontrado no índice 8!
\end{itemize}
\end{frame}

%-------------------------------------------------------
\begin{frame}\frametitle{Pesquisa Binária: Exemplo (\texttt{valor = 5})}
\begin{itemize}

	\item Passo 1:
\begin{tabular}{cccccccccc}
\tiny{0} & \tiny{1} & \tiny{2} & \tiny{3} & \tiny{4} & \tiny{5} & \tiny{6} & \tiny{7} & \tiny{8} & \tiny{9}\\
\hline
\multicolumn{1}{|c|}{2} & \multicolumn{1}{c|}{4} & \multicolumn{1}{c|}{6} & \multicolumn{1}{c|}{8} & \multicolumn{1}{c|}{10} & \multicolumn{1}{c|}{12} & \multicolumn{1}{c|}{14} & \multicolumn{1}{c|}{16} & \multicolumn{1}{c|}{18} & \multicolumn{1}{c|}{20}\\
\hline
$\uparrow$ & & & & $\uparrow$ & & & & & $\uparrow$ \\
{\tiny\texttt ini} & & & & {\tiny\texttt meio} & & & & & {\tiny\texttt fim} \\
\end{tabular}\\~\\~

	\item Passo 2:
\begin{tabular}{cccccccccc}
\tiny{0} & \tiny{1} & \tiny{2} & \tiny{3} & \tiny{4} & \tiny{5} & \tiny{6} & \tiny{7} & \tiny{8} & \tiny{9}\\
\hline
\multicolumn{1}{|c|}{2} & \multicolumn{1}{c|}{4} & \multicolumn{1}{c|}{6} & \multicolumn{1}{c|}{8} & \multicolumn{1}{c|}{10} & \multicolumn{1}{c|}{12} & \multicolumn{1}{c|}{14} & \multicolumn{1}{c|}{16} & \multicolumn{1}{c|}{18} & \multicolumn{1}{c|}{20}\\
\hline
$\uparrow$ & $\uparrow$ & & $\uparrow$ & & & & & & \\
{\tiny\texttt ini} & {\tiny\texttt meio}& & {\tiny\texttt fim} & & & & & & \\
\end{tabular}

\end{itemize}
\end{frame}

%-------------------------------------------------------
\begin{frame}\frametitle{Pesquisa Binária: Exemplo (\texttt{valor = 5})}
\begin{itemize}

	\item Passo 3:
\begin{tabular}{cccccccccc}
\tiny{0} & \tiny{1} & \tiny{2} & \tiny{3} & \tiny{4} & \tiny{5} & \tiny{6} & \tiny{7} & \tiny{8} & \tiny{9}\\
\hline
\multicolumn{1}{|c|}{2} & \multicolumn{1}{c|}{4} & \multicolumn{1}{c|}{6} & \multicolumn{1}{c|}{8} & \multicolumn{1}{c|}{10} & \multicolumn{1}{c|}{12} & \multicolumn{1}{c|}{14} & \multicolumn{1}{c|}{16} & \multicolumn{1}{c|}{18} & \multicolumn{1}{c|}{20}\\
\hline
 & & $\uparrow$ & $\uparrow$ & & & & & & \\
 & & {\tiny\texttt ini} & {\tiny\texttt fim} & & & & & & \\
 & & {\tiny\texttt meio} & & & & & & & \\
\end{tabular}\\~\\~

	\item Passo 4:
\begin{tabular}{cccccccccc}
\tiny{0} & \tiny{1} & \tiny{2} & \tiny{3} & \tiny{4} & \tiny{5} & \tiny{6} & \tiny{7} & \tiny{8} & \tiny{9}\\
\hline
\multicolumn{1}{|c|}{2} & \multicolumn{1}{c|}{4} & \multicolumn{1}{c|}{6} & \multicolumn{1}{c|}{8} & \multicolumn{1}{c|}{10} & \multicolumn{1}{c|}{12} & \multicolumn{1}{c|}{14} & \multicolumn{1}{c|}{16} & \multicolumn{1}{c|}{18} & \multicolumn{1}{c|}{20}\\
\hline
 & $\uparrow$ & $\uparrow$ & & & & & & & \\
 & {\tiny\texttt fim} & {\tiny\texttt ini} & & & & & & & \\
\end{tabular}
 ~ Valor 5 \textbf{NÃO} encontrado!

\end{itemize}
\end{frame}

%-------------------------------------------------------
\begin{frame}[fragile]\frametitle{Pesquisa Binária Recursiva}
\lstinputlisting[basicstyle=\ttfamily\scriptsize]{src/pesquisaBinariaRec.cpp}
\end{frame}

%=======================================================
\section{Exercícios}

%-------------------------------------------------------
\begin{frame}\frametitle{Exercícios}
\begin{enumerate}
	\item Dado um valor inteiro e positivo (\texttt{n}), o valor da constante de \emph{Euler} poder ser calculando com precisão diretamente proporcional a \texttt{n} através da fórmula:
\[ E = \frac{1}{0!} + \frac{1}{1!} + \frac{1}{2!}+ \frac{1}{3!} + ... + \frac{1}{n!}\]
Implemente, em C, um método recursivo que recebe \texttt{n}, retornando o valor de \emph{Euler} calculado usando a fórmula acima.
	\item Considere a função abaixo, que implementa o somatório de um vetor e implemente a sua versão recursiva.
\lstinputlisting[basicstyle=\ttfamily\tiny]{src/somatorio.cpp}
\end{enumerate}
\end{frame}

%-------------------------------------------------------
\begin{frame}\frametitle{Exercícios}
\begin{enumerate}
        \setcounter{enumi}{2}
	\item Considere a função que implementa o algoritmo \emph{Bubble Sort} (apresentada na unidade anterior) e implemente a sua versão recursiva).
	\item Implemente uma função recursiva que inverte os elementos de um vetor, trocando o primeiro elemento com o último, o segundo com o penúltimo, e assim sucessivamente.
	\item Implemente uma função recursiva que verifica se uma cadeia de caracteres recebida como parâmetro é um palíndromo ou não. Por exemplo, ``socorrammesubinoonibusemmarrocos'' é um palíndromo.
	%\item Faça o algoritmo da potência de forma não recursiva e de forma recursiva. Considere que o valor da base e do expoente são recebidos por parâmetro, inteiros e positivos.
\end{enumerate}
\end{frame}

%=======================================================
\section{Soluções}

%-------------------------------------------------------
\begin{frame}\frametitle{Soluções}
\begin{enumerate}
	\item \lstinputlisting[basicstyle=\ttfamily\tiny]{src/eulerRec.cpp}

	\item \lstinputlisting[basicstyle=\ttfamily\tiny]{src/somatorioRec.cpp}

	\item \lstinputlisting[basicstyle=\ttfamily\tiny]{src/bubbleSortRec.cpp}
\end{enumerate}
\end{frame}

%-------------------------------------------------------
\begin{frame}\frametitle{Soluções}
\begin{enumerate}
        \setcounter{enumi}{3}
	\item \lstinputlisting[basicstyle=\ttfamily\tiny]{src/inverteVetorRec.cpp}

	\item \lstinputlisting[basicstyle=\ttfamily\tiny]{src/palindromoRec.cpp}

\end{enumerate}
\end{frame}

\end{comment}

%=======================================================
\section{Créditos}

%-------------------------------------------------------
\begin{frame}\frametitle{Créditos}
\begin{itemize}
	\item Estas lâminas contêm trechos adaptados de materiais criados e disponibilizados pela professora Isabel Harb Manssour [*].
\end{itemize}
\end{frame}

%-------------------------------------------------------
\end{document}

